\documentclass[9pt, reqno]{amsart}

\usepackage[a4paper,twoside,left=30mm,right=30mm,top=40mm,bottom=35mm]{geometry}
\usepackage{amsmath}                
\usepackage{amsthm}                
\usepackage{amssymb}
\usepackage[foot]{amsaddr}
\usepackage{dcolumn}     
\usepackage{multirow}                                                                
\usepackage{graphicx}      
\usepackage[usenames, dvipsnames,svgnames,table]{xcolor}
\usepackage[small]{caption}  
\usepackage{tcolorbox}
\usepackage{mathrsfs}                                   
\usepackage{longtable}                                  
\usepackage{lscape}                                     
\usepackage{afterpage}                                  
\usepackage[figuresleft]{rotating}                     
\usepackage{setspace} 
\definecolor{myGreen}{HTML}{7d9b76}		
\definecolor{myBrown}{HTML}{4a3139}		
\definecolor{myBlack}{HTML}{2B2B2B}			
\definecolor{myWhite}{HTML}{E5E6E4} 
\definecolor{myGrey}{HTML}{9B9B9B} 
\usepackage[hidelinks, colorlinks=True, urlcolor=myGreen, citecolor=myBrown, linkcolor=myBrown]{hyperref}                             
\usepackage{pdfpages}
\usepackage{subfiles}
\usepackage{csquotes}
\usepackage{wrapfig}
\usepackage{subcaption}
\usepackage{physics}
\usepackage{lipsum}
\usepackage{mathtools}
\usepackage{booktabs}
\usepackage{arev}
\usepackage[T1]{fontenc}
\usepackage{floatrow}
\usepackage{svg}
\usepackage{enumitem}
\usepackage{arydshln}	
\usepackage{etoolbox} 
\usepackage{lipsum}
\usepackage{titlesec}
\usepackage[title,toc]{appendix}
\usepackage{abstract}
\usepackage[numbers]{natbib}
\usepackage{dashrule}

\usepackage{tikz}
\usetikzlibrary{positioning, arrows.meta, decorations.markings, calc, intersections}
\tikzset{arrowmark/.style={postaction={decorate, decoration={markings, mark=at position #1 with {\arrow{>};}}}},
    arrowmark/.default={.5}
} % configures ability to position in front etc...

\renewcommand{\footnotesize}{\fontsize{8pt}{11pt}\selectfont}
\renewcommand{\abstractnamefont}{\normalfont\footnotesize\bfseries}
\renewcommand{\abstracttextfont}{\normalfont\footnotesize}

\setlength{\parindent}{0em} 				
\setlength{\headheight}{15pt}		
\renewcommand{\arraystretch}{2}
\numberwithin{equation}{section}								
\floatsetup[subfigure]{capbesideposition={left, center}}				

\newcommand{\distribution}[3]{\mathrm{#1}(#2,#3)}
\renewcommand{\(}{\left(}
\renewcommand{\)}{\right)}
\renewcommand{\vec}[1]{\mathbf{#1}}
\newcommand{\kmsquared}{\text{km}^2}						
\newcommand{\hasquaredkm}{\rho\text{ha}/\kmsquared}
\renewcommand{\abs}[1]{\mathrm{abs}\(#1\)}

\newtheorem{theorem}{Theorem}[section]
\newtheorem{lemma}{Lemma}[section]
\newtheorem{corollary}{Corollary}[section]
\newtheorem{proposition}{Proposition}[section]
\theoremstyle{definition}
\newtheorem{definition}{Definition}[section]
\theoremstyle{definition}
\newtheorem{example}{Example}[section]
\theoremstyle{definition}
\newtheorem*{remark}{Remark}

\newcommand{\boxtitle}[1]{
	\colorbox{myBrown!50}{%
		\color{myBlack}
		\transparent{1} %
		\enspace \begin{minipage}[c]{\linewidth-5.4\fboxsep}
			\vspace*{.25em} #1 %
		\end{minipage}%
	}
}
%
\newcommand{\boxsubtitle}[1]{
    \colorbox{myGreen!50}{%
		\color{myBlack}
		\transparent{1} %
		\enspace \begin{minipage}[c]{\linewidth-4.85\fboxsep}
			\vspace*{.25em} #1 %
		\end{minipage}%
	}
}
%
\newcommand{\boxsubsubtitle}[1]{
		\color{myBlack}
		\transparent{1} %
		\begin{minipage}[c]{\linewidth}
            % \color{myBrown}{\rule{\linewidth}{1pt}} 
			#1 \\ %
            % \vspace{-.5em}
            \color{myBrown}{\rule{\linewidth}{1pt}}
		\end{minipage}%
}

\titleformat{\section}{\normalfont\Large\bfseries\boxtitle}{\thesection}{2em}{\raggedright}
\titlespacing*{\section}{0em}{1em}{1em}

\titleformat{\subsection}{\normalfont\bfseries\boxsubtitle}{\thesubsection}{2em}{\raggedright}
\titlespacing*{\subsection}{0em}{1em}{1em}

\titleformat{\subsubsection}{\normalfont\itshape\color{myBrown}}{\thesubsubsection}{2em}{}
\titlespacing*{\subsubsection}{0em}{0em}{.5em}

\makeatletter
\renewcommand{\@biblabel}[1]{[#1]\hfill}
\makeatother

\makeatletter
\patchcmd{\@maketitle}{\normalsize}{\Large}{}{} 
\makeatother

\makeatletter
\patchcmd{\@setemails}{E-mail addresses}{Corresponding author}{}{} 
\makeatother    

\definecolor{draftcol}{HTML}{FF0000}
\newcommand{\draft}[1]{\textcolor{draftcol}{#1}}

\newcommand\pmeq{\stackrel{\mathclap{\mbox{$\pm$}}}{=}}


\begin{document}
%%%%%
%	TITLE
%%%%%
\title{Determining the expansion rate of an invasive tree pest}

\author[JP McKeown]{Jamie P McKeown\textsuperscript{1}}
\address[A1]{School of Mathematics, Statistics, and Physics, Newcastle University, Newcastle upon Tyne, UK.}
\email[A1]{j.p.mckeown2@ncl.ac.uk}

\date{\today}
\maketitle 

\begin{abstract}
 \centering \bigskip
  \begin{minipage}{\dimexpr\paperwidth-10cm}
 	\draft{\lipsum[1]}  \\
	
 	\bigskip
 	\noindent\textbf{Keywords:} \draft{Key words}\\
	
 	\bigskip
 	\noindent\textbf{MSC Classification:} \draft{Classification} \\	
 \end{minipage} 
 \end{abstract} 




%%%%%
%	CONTENT
%%%%%
\section*{Some ideas}

	\subsection*{Misc notes} 
	We can consider the following:
                \begin{itemize}
                    \item Two models: M1 constant-coefficient model. 
                    \item Inference for "both" models - summarise paper 1 results for homogeneous model. / or maybe just pick some parameters and leave inference for follow up paper?
		              \begin{itemize}
		                  \item Compare models using the DIC 
		              \end{itemize}
		              \begin{itemize}
		                  \item Posterior predictives for both? Using half the number of test sites (4 each - 2023, 2019, 2015, 2010)
		              \end{itemize}
                    \item Approaches for calculating the speed of propagation:
                        \begin{enumerate}
                            \item Convex hull decomposition and geodesic distance (heat method): using supremum/maximum of distance to hull vertices.
                            \item Tanh fit and midpoint trajectory: again, using maximum 
                            % \item Also compare average distance to get speed.
                            \item 8-point compass of speeds: using maximum along given directions (for convex hull, use $\frac{\pi}{8}$ radian arcs, for tanh fit, use the specific cross sections $(y=0)^+, (y=0)^-, (x=0)^+, (x=0)^-, (y=x)^+, (y=x)^-, (y=-x)^+, (y=-x)^-$ )
                            \item For each of these two approaches, we will measure distance from two sources: a single point source (the centre of the initial state) and from the (edge of the) initial territory we obtain from our circle fit. For CH/GD, this defines a (generalized) Dirac function that's fed directly into the heat method. For tanh fit, single point - measure distance from that point; initial state - measure distance from centre and subtract radius of circle.
                        \end{enumerate}
                \end{itemize}
	
	

	\subsection*{Methods for determing the position of the front}
	\begin{enumerate}
		\item Convex hull. 	
		\item Tanh fit along one-dimensional cross sections
		\item Circle fit (a la Suprunenko)

	\end{enumerate}

	\subsection*{Methods for determing rate of expansion/propagation}
	\begin{enumerate}
		\item Geodesic distances via the heat method  / linear rate of spread a la Mineur et al 2010
		\item Measuring distance traversed by midpoint of the tanh fit? / linear rate of spread a la mineur et al 2010
		\item Maximum distance method (and potentially others quoted in Suprunenko et al 2021 / Preuss et al 2014
		\item 95th gamma quantile (Preuss et al 2014)
		\item Infested area? a la Hill et al 2001 / accumulation of occupied grid squares a la Mineur et al 2010
	\end{enumerate}





%%%%%%%%%%%%%%%%%%%%%
%		 	BIBLIO		     	%
%%%%%%%%%%%%%%%%%%%%%
\def\bibcommenthead{}%
\bibliographystyle{sn-mathphys-num}
\bibliography{references}

\end{document} 
\typeout{get arXiv to do 4 passes: Label(s) may have changed. Rerun}